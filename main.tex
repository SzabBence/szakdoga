\documentclass{article}
\usepackage[utf8]{inputenc}
\usepackage{algorithm}
\usepackage{algorithmic}
\usepackage{amsfonts}
\usepackage{graphicx}
\usepackage{listings}
\usepackage{xcolor}
\usepackage{amsmath}
\usepackage{amsthm,amssymb}
\theoremstyle{definition}
\theoremstyle{theorem}
\usepackage{geometry}
\geometry{a4paper,left=25mm,right=25mm,bottom=25mm,top=25mm,footskip=10mm}
\newtheorem{definition}{Definíció}
\newtheorem{theorem}{Tétel}

\newtheorem{example}{Példa}
\renewcommand{\qedsymbol}{\rule{0.7em}{0.7em}}
\title{Szakdolgozat}
\author{Szabó Bence Dániel }
\date{Date}

\definecolor{codegreen}{rgb}{0,0.6,0}
\definecolor{codegray}{rgb}{0.5,0.5,0.5}
\definecolor{codepurple}{rgb}{0.58,0,0.82}
\definecolor{backcolour}{rgb}{0.95,0.95,0.92}

\lstdefinestyle{mystyle}{
    backgroundcolor=\color{backcolour},   
    commentstyle=\color{codegreen},
    keywordstyle=\color{magenta},
    numberstyle=\tiny\color{codegray},
    stringstyle=\color{codepurple},
    basicstyle=\ttfamily\footnotesize,
    breakatwhitespace=false,         
    breaklines=true,                 
    captionpos=b,                    
    keepspaces=true,                 
    numbers=left,                    
    numbersep=5pt,                  
    showspaces=false,                
    showstringspaces=false,
    showtabs=false,                  
    tabsize=2
}
\lstset{style=mystyle}
\begin{document}

\begin{center}
\fontsize{40pt}{12pt}\selectfont
    
    DIFFERENCIÁLEGYENLETEK NUMERIKUS MEGKÖZELÍTÉSE A KÖZGAZDASÁGBAN\\
    \bigskip
    Témavezető: Kánnai Zoltán\\
    \bigskip
    Készítette: Szabó Bence Dániel\\
    \bigskip
    Szak : Gazdaság-és pénzügy matematikai elemzés
\end{center}

\pagebreak
\tableofcontents
\section{Bevezetés}
A szakdolgozatomban megvizsgálom a differenciálegyenletek numerikus megoldásait. Ehhez kimondom a szükséges tételeket, definíciókat és módszereket. A gyakorlati megvalósítást pedig a Python programozási nyelvben valósítom meg. Azonban elkerülhetetlen a programozás alatt, hogy az átláthatóság érdekében kövessünk bizonyos normákat, így ezeket itt legjobb tudásom szerint követem. A dolgozat felépítése egy Python nyelv bevezetővel indít, majd tárgyalja a numerikus integrálás témakörét. Itt szándékosan kerülöm a különböző módszerek hibatulajdonságait, mivel ez önmagában kitenne egy szakdolgozatot. A dolgozatban beépített integrál függvényt használok a $\textit{SciPy}$ csomagból. Ezután tárgyalom az ODE-k témakörét.\\

Több rész egyelőre nem készült el a szakdolgozatból.\\ Tervek szerint folytatom a módszertanok tárgyalását és utána konkrét modelleken is alkalmazom azokat

\section{Python nyelv}
\begin{center}
    \includegraphics[width=\textwidth]{plots/python-logo.png}
    %https://www.python.org/static/community_logos/python-logo-master-v3-TM.png
\end{center}

A Python egy magas szintű programozási nyelv egy könnyen elsajátítható szintaxissal. Azért erre a nyelvre esett
a választásom, mert sok csomag elérhető hozzá, nagyon jól dokumentált, nem kell típusokat használni benne az 
esetek nagy többségében, nyílt és ebben dolgozom, ezért ezt ismerem a legjobban. 
\subsection{Python függvények}
Pythonban minden függvényt a \textit{def} szóval kezdünk, amit a függvény neve követ, majd zárójelek közé beírjuk a paramétereket és vesszővel választjuk el, ha vannak. A csukó zárójel után mindig ki kell tenni a kettőspontot. Ezután a következő sorban kell folytatnunk a kódot, de egy tabulátornyi hellyel jobbrább, mint ahogy a \textit{def} szócskát elhelyeztük.
\lstinputlisting[language=Python]{python-bevezetohoz-scriptek/function_template.py}
A függvény paramétereinél nem szükséges, hogy megadjuk milyen típusú inputot vár el, azonban a saját segítségünkre megadhatjuk ezeket a következő képpen: a paraméter neve után kettőspontot írunk, majd az adattípus nevét, pl. str, mint String, dict, mint Dictionary, int, mint integer, stb...
\lstinputlisting[language=Python]{python-bevezetohoz-scriptek/typing_template.py}

A függvény visszatérési értékét pedig a \textit{return} szó után adhatjuk meg, amennyiben szükség van rá
\subsection{Ciklusok}
\subsubsection{For ciklus}
A \textit{for} ciklus ebben a nyelvben nagyon hasonlít más nyelvek \textit{for each} megoldásaira, mert a for mindig valamilyen listán iterál végig, legyenek azok számok vagy egyéb objektumok. A szintaxisa a következő :
A \textit{for} szó után megadjuk a változót, amit használunk az iterálás alatt, majd az \textit{in} szó után megadjuk a sorozatot, amin végiglépkedünk. A végére itt is kettőspontot kell raknunk, akárcsak a függvényeknél

\lstinputlisting[language=Python]{python-bevezetohoz-scriptek/for_template.py}

\subsubsection{While ciklus}
A while ciklust hasonlóan adjuk meg , mint más nyelvekben. A \textit{while} szó után jön az eldöntendő feltétel, és ameddig a feltétel igaz, addig a blokkban lefog futni a kódrészlet. A feltétel végét itt is kettősponttal zárjuk le.
\lstinputlisting[language=Python]{python-bevezetohoz-scriptek/while_template.py}

\subsection{Python csomagok}
Mint minden nyelvben, itt is tudunk importálni már készen levő csomagokat, amelyek tartalmaznak a célnak megfelelő osztályokat, azon belül metódusokat. Ha nincs feltelepítve a csomag, akkor a \textbf{pip install csomag-neve} paranccsal tehetjük meg azt.
Amint ezt megtettük, az \textit{import csomag-neve as alias-csomag-neve} módon importálhatjuk a kódunkba. A könnyebb olvashatóság érdekében szokás \textit{alias}-t használni, de nem szükséges.

\lstinputlisting[language=Python]{python-bevezetohoz-scriptek/import_template.py}
\section{Numerikus integrálás}
A numerikus integrálás nagyon fontos része a numerikus módszereknek. Egyrészről gyorsabban megkaphatjuk a határozott  integrál értékét, másrészről vannak olyan függvények, amelyeknek nem tudjuk kiszámolni papíron az integrál értéket, csak becsülni tudjuk alulról és/vagy felülről. A numerikus integrálás nagyban támaszkodik a kvadratúra képletekre.
\begin{definition}
Kvadratúra képletek általánosan

Az integrál értékét tudjuk közelíteni az ún. \textit{Kvadratúra képlettel}.\newline 
\begin{equation*}
    \int_{a}^{b} f(x) dx \approx \sum_{k = 0}^{n} c_k f(x_k) = \sum_{k=0}^n \sigma_k \;\; , ahol\;\; x_i \in [a,b]
\end{equation*}

\end{definition}

Most nézzünk meg néhány kvadratúra képletet.
\subsection{Newton - Cotes formula}


Az integrálni való halmazt osszuk fel egyforma hosszúságúra, így vegyünk ekvidisztáns alappontokat.

\begin{equation*}
    x_k = a + hk \;\;, \;\;ahol \;\;k=0,1,...,n ,\;\; h = \frac{b-a}{n}
\end{equation*}
Írjuk fel az interpolációs kvadratúra képletet.
\newline
\begin{equation*}
    c_k = \int_a^{b} l_k(x) dx = h \int_0^{n} l_k (a+ht) dt = \frac{b-a}{n} \int_0 ^{n} \Pi_{j=0, j \neq k}^n \frac{t-j}{k-j} dt = \frac{b-a}{n} \frac{1}{k! (n-k)!} \int_0^n \Pi_{j=0}^{k-1} (t-j) \Pi_{j=k+1}^{n} (j-t) dt
\end{equation*}

\subsection{Összetett kvadratúra képlet}

\begin{equation*}
 \int_a^{b} f(x) dx = \sum_{j = 1} ^{m} \int_{a_{j-1}} ^{a_j} f(x) dx 
\end{equation*}
 ahol $a_0 = a, a_m =b $  

A baloldali részintegrált írjuk fel az interpolációs kvadratúra képlettel:
\begin{equation*}
\int_{a_{j-1}}^{a_j} f(x) dx \approx \sum_{k=0}^{r} c_{k,j}f(x_{k,j})
\end{equation*}

Innen helyettesítsünk vissza az eredeti képletbe
\begin{equation*}
\int_a^b f(x) dx \approx \sum_{j=1}^{m} \sum_{k=0}^{r} c_{k,j} f(x_{k,j})
\end{equation*}


\subsection{Érintő formula}

Az érintő formulához felfogjuk használni a függvények lineáris Taylor közelítését. Ezen felül vegyük az intervallum ekvidisztáns felosztását.
\begin{equation*}
   x_k = a + (k - \frac{1}{2})h ,\;\; ahol\;\; h = \frac{b-a}{n},\;\; k =1,...,n. 
\end{equation*}



A Taylor közelítése a függvénynek a $[x_k-\frac{h}{2},x_k+\frac{h}{2}]$ intervallumon
\begin{equation*}
    f(x) \approx f(x_k)+f'(x_k)(x-x_k).
\end{equation*}

\begin{equation*}
\int_{x_k-\frac{h}{2}}^{x_k + \frac{h}{2}} \approx \int_{x_k - \frac{h}{2}}^{x_k + \frac{h}{2}} [f(x_k) + f'(x_k)(x-x_k)] dx = hf(x_k)
\end{equation*}
\newline
Innen pedig kapjuk, hogy
\begin{equation*}
\int_a^b = f(x) dx = \sum_{k=1}^n \int_{x_k - \frac{h}{2}}^{x_k + \frac{h}{2}} f(x)dx \approx h \sum_{k=1}^n f(x_k)
\end{equation*}



\subsection{A trapéz módszer és a Newton - Cortes módszer}
Írjuk fel a Newton - Cortes formulát n = 1 és k = 0 melett:
\begin{equation*}
c_0 = \frac{b-a}{1} \frac{1}{0!(1-0)!} \int_0^{1} \Pi_{j=0}^{0-j} \Pi_{j=0+1}^{1} (j-t) dt = (b-a) \int_0^1 (1-t) dt = \frac{b-a}{2}
\end{equation*}

\begin{equation*}
c_1 = \frac{b-a}{2}
\end{equation*}
Ekkor 

\begin{equation*}
    \int_a^b f(x) dx \approx \frac{b-a}{2}[f(a) + f(b)] = \sum_{k=0}^{1} \frac{b-a}{2} f(x_k),\\
    \;\;ahol \;\;x_0 = a,\;\;x_1=b
    \;\;
\end{equation*}

Ez pedig pont a trapéz területe! Nézzünk erre egy példát :
\newline

\includegraphics{plots/trapez_with_area.png}

Az integrál értéke ezen az intervallum 7/3. 
\begin{center}
$\int_{1}^{2} x ^2 dx = \frac{7}{3}$
\end{center}
Azonban, ha ezen paraméterek között végezzük el a trapéz - területszámítást akkor 2.5 - t kapunk eredményként. Ezt onnan is láthatjuk, hogy a fenti ábrán a kék és piros vonal között még akad zöld terület. Azonban most számoljuk ki ezzel a módszerrel az integrál értékét az [1,1.5] és [1.5,2] intervallumokon. Ekkor eredményül 0.8125 és 1.5625, összesen 2.375 értéket kapunk, ami közelebb van a 7/3-hoz

\includegraphics{plots/trapez_n=2.png}

\subsection{Python kódok a fejezethez}
A trapéz függvényben használjuk a globals() metódust. Ez a metódus visszaadja az összes olyan objektumot, amely az adott programunk futásakor belekerült a compilerbe(fordítoprogram). Ezek között lesznek azok a függvények, amiket megírtunk. A globals() egy dictionaryt ad vissza, amiben mi a függvény nevét fogjuk használni, mint kulcs.
\pagebreak
\lstinputlisting[language=Python]{functions.py}

\section{Szukcesszív approximációs módszer}

\begin{definition}{ODE}\\
Közönséges differenciálegyenletnek(Ordinary Differential Equation) nevezzük azokat az egyenleteket, amelyekben szerepel egy függvény egy változója szerinti deriváltja és célunk meghatározni a derivált függvényt.\\
Például
\begin{equation*}
    \frac{dx}{dt} (t) = cos(t)
\end{equation*}
\end{definition}

\begin{definition}Autonóm differenciálegyenlet\\
Írjuk fel a következő ODE-t:
\begin{equation*}
    f(x,y) = \frac{dy}{dt} = y'
\end{equation*}
Azokat a differenciálegyenleteket nevezzük autonómnak, amelyek nem függnek az x változótol a fent leírt egyenletben. Erre egy jó példa lehet a következő feladat. 
\begin{equation*}
    \frac{dy}{dt} = f(y)
\end{equation*}
Ha t paramétert az időnek vesszük, akkor ebben az egyenletben azt láthatjuk, hogy nem függ az időtől az y megváltozása, mindig állandó a deriváltja, f(y) a változás időtől függetlenül.
\end{definition}

\begin{definition}Lineáris differenciálegyenlet\\
Lineáris differenciálegyenletnek nevezzük azokat a differenciálegyenleteket, ahol a jobb oldali f függvényt feltudjuk írni x deriváltjainak lineáris kombinációjaként.
Példa erre a következő:
\begin{equation*}
    a_1 x'(t) + a_2 x''(t) = f(t,x(t))
\end{equation*}
\end{definition}

\begin{definition}Nemlineáris differenciálegyenlet\\
Azokat a differenciálegyenleteket nevezzük \textit{nemlineáris differenciálegyenleteknek}, amelyben a deriváltak nem írhatóak fel egy lineáris kombinációban
\end{definition}

\begin{definition}Homogén differenciálegyenlet\\
Az olyan differenciálegyenlet, amely 0-val egyenlő, \textit{homogén differenciálegyenleteknek} nevezzük.
\end{definition}
\begin{definition}Nemhomogén differenciálegyenlet\\
Az olyan differenciálegyenleteket, amelyek nem 0-val egyenlőek \textit{nemhomogén differenciálegyenleteknek} nevezzük
\end{definition}
\begin{definition}{Kezdetiérték-feladat}\\
Vegyünk egy ODE problémát és adjunk meg hozzá egy (kezdeti) pont-függvény párost. Az ODE-t ezzel a tulajdonsággal ellátva kezdetiérték problémának  nevezzük
\begin{equation*}
    \begin{cases}
       x'(t) = f(t,x(t))\\
       x(t_0) = x_0 \\
    \end{cases}       
\end{equation*}
\end{definition}
\begin{definition}{Lipschitz folytonosság}

\begin{equation*}
Legyen \;\;    f : \mathbb{R} \rightarrow \mathbb{R}, \; és \;\; L\; \;\in \mathbb{R}^+
\end{equation*}

Ha f minden x és y pontjára fennáll a következő 
\begin{equation*}
    |f(x) - f(y)| \leq L |x-y|
\end{equation*}
egyenlőtlenség, akkor \textit{Lipschitz folytonosnak} nevezzük az f függvényt
\end{definition}
\begin{definition}{Kontrakció}\\
Legyen (X,$\textit{d}$) egy teljes metrikus tér. T leképezést $\textit{kontrakciónak}$ nevezzük X alaphalmazon, ha $\exists$ $q \in [0,1)$, hogy 
\begin{equation*}
    d(T(x),T(y)) \leq q \textit{d}(x,y) \;\;,\forall x,y \in X
\end{equation*}
Pongyolán fogalmazva ezt jelenti, hogy azok a  Lipschitz folytonos függvények a kontrakciók, amelyekben az L Lipschitz konstans kisebb 1-nél 
\end{definition}
\begin{theorem}[Banach-féle fixponttétel]
Legyen (X,$\textit{d}$) egy nemüres, teljes metrikus tér ellátva egy T kontrakcióval. Ekkor 
\begin{equation*}
    \exists!\;\; x^* \in X : T(x^*) = x^*
\end{equation*}
\end{theorem}
\begin{proof}
Részletesen nem tisztázzuk a tétel bizonyítását, csak megemlítjük a főbb lépéseket.
 
\begin{equation*}
Legyen\;\;(X_n)_{n \in \mathbb{N}},\;\; x_n = T(x_{n-1}), \;\;x_0 \in X \;\;egy \;\; sorozat
\end{equation*}
Mivel T kontrakció, így igaz a következő egyenlőtlenség 
\begin{equation*}
    d(x_{n+1},x_n) \leq q^n d(x_1,x_0)
\end{equation*}
Ha ezeket felhasználva elkezdünk \textit{n}-ben iterálni, könnyű belátni, hogy
\begin{equation*}
    \forall \;\; \epsilon >0 : \;\;d(x_j,x_k) j < \epsilon\;\;, \;\;j\neq k  \;\; 
\end{equation*}
azaz  $(x_n)$ Cauchy  sorozat.
Mivel (X,d) metrikus tér teljesn ezért a sorozatunknak létezik határértéke, sőt ez a határéték pont T leképezés fix pontja lesz. A fix pont egyértelműségét igazolhatjuk, ha indirekt feltesszük, hogy van másik fix pont.
\end{proof}
%https://en.wikibooks.org/wiki/Ordinary_Differential_Equations/The_Picard–Lindelöf_theorem
\begin{theorem} [Picard - Lindenlöf] \\
Legyen I =[a,b] egy intervallum , és egy f függvény,

\begin{equation*}
f :  I \times \mathbb{R}^n \rightarrow \mathbb{R}^n 
\end{equation*}
Legyen x' pedig egy ODE
\[
x'(t) = f(t,x(t))
\]
Ha f Lipschitz folytonos x(t)-ben, akkor az ODE-nek létezik egyértelmű megoldása [a,a + $\epsilon$] 
intervallumon minden 
\[
x(0) = x_0 \in \mathbb{R}^n, \epsilon < \frac{1}{L}
\] halmazra és L pedig x(t) Lipschitz konstansa
\end{theorem}
\begin{proof}
Írjuk fel ezt egy kezdeti-érték feladatra 
\begin{equation*}
    \begin{cases}
       f :  I \times \mathbb{R}^n \rightarrow \mathbb{R}^n , t \in [a,a + \epsilon]\\
       x'(t) = f(t,x(t))\\
    \end{cases}       
\end{equation*}

Ez a Newton - Leibniz tétel miatt pedig erre az alakra hozható:
\begin{equation*}
    \forall t \in [a, a +\epsilon] : x(t) = x_0 + \int_{a}^{t} f(s,x(s)) ds
\end{equation*}
 ahol pedig $\epsilon < \frac{1}{L}$. Eszerint x(t) fixpontja a következő függvénynek : 
 
 \begin{equation*}
     T : \mathcal{C}([a, a +\epsilon]) \rightarrow \mathcal{C}([a, a +\epsilon]), T(x,t) := x_0 + \int_a^t f(s,x(s)) ds
 \end{equation*}
 T kielégíti a Lipschitz folytonosságot, mivel
 \begin{equation*}
    \begin{split}
     \lVert T(x,t) - T(y,t) \rVert =\lVert \int_a^t f(s,x(s)) ds - \int_a^t f(s,x(s)) ds \rVert &
     \leq \int_a^t \lVert f(s,x(s)) - f(s,y(s)) \rVert ds \leq \int_a^t L \lVert x(s) - y(s) \rVert &
     \leq (t-a) L \lVert x-y\rVert_{\infty} \leq \epsilon L \lVert x - y \rVert_{\infty}
     \end{split}
 \end{equation*}
 
 Mivel T pedig kontrakció, így a Banach-féle fixpont tétel alkalmazható,amivel igazoltuk a létezését a megoldásnak. Továbbá felhasználva a Peano-egyenlőtlenséget megkapjuk az egyértelműséget is.
\end{proof}
\begin{definition}[Szukcesszív approximációs módszer]
Vegyük a következő kezdetiérték problémát
\begin{equation*}
    \begin{cases}
       x'(t) = f(t,x(t))\\
       x(a) = b
    \end{cases}       
\end{equation*}
Fentebb tárgyaltuk a Picard-Lindelöf tételnél, hogy a Newton - Leibniz tétel miatt fennáll a következő
egyenlőség: \newline
\begin{equation*}
    x(t) = x(0) + \int_a^t f(s,x(s)) ds
\end{equation*}
Ekkor vegyük az \[x_0(t) = b\] állandót, majd kezdjünk el előre iterálni a következő képlet alapján:
\begin{equation*}
    x_{n + 1}(t) = b + \int_a^t f(s,x_n(s)) ds 
\end{equation*}
\end{definition}

\begin{example}[Exponenciális függvény]
\begin{equation*}
    \begin{cases}
       y'(t) = y\\
       y(0) = 1
    \end{cases}       
\end{equation*}
Kezdjünk el lépkedni.
\begin{equation*}
    y_1(t) = 1 + \int_0^x 1 dt = 1 + x
\end{equation*}
\begin{equation*}
    y_2(t) = 1 + \int_0^x 1+t dt = 1 + x + \frac{x^2}{2}
\end{equation*}
\begin{equation*}
    y_3(t) =  1 + x + \frac{x^2}{2} + \frac{x^3}{6}
\end{equation*}
\begin{equation*}
    y_4(t) = 1 + x + \frac{x^2}{2} + \frac{x^3}{6} + \frac{x^4}{24}
\end{equation*}
\begin{equation*}
    y_{\infty}(t) = exp(x)
\end{equation*}
\end{example}
\subsection{Python kódok a fejezethez}
\lstinputlisting[language=Python]{fokozatos_kozelitesek/successive.py}
\section{Runge-Kutta módszer}
A Runge-Kutta módszerek az egyik legelterjedtebb megoldó módszerek a numerikus analízis eszköztárában, mivel 'olcsó' a használatuk, nem igényelik a függvények deriváltjainak az ismeretét és mégis elfogadható pontossággal műküdnek (ez például nem mondható el feltétlen az Euler módszerről). A módszer a nevét két német matematikusról kapta, nevesen Carl Runge és Wilhelm Kuttáról.

\subsection{Euler módszer}
Alapvetően nem fogjuk használni az Euler módszert, azonban mindképp említést kell tennünk róla, mivel a Runge - Kutta módszer erre épít. Ez nagyon szemléletes megoldási metódus, azonban nem praktikus ha általánosan akarunk felírni egy kódot hozzá, mert épít a deriváltakra.\\

Vegyünk egy kezdetiérték problémát és hozzá egy lépésközt.
\begin{equation*}
    \begin{cases}
       x'(t) = f(t,x(t))\\
       x(a) = b \\
       t_i = a + ih , i \in \{0,1,...,N\}
    \end{cases}       
\end{equation*}

Itt pedig a lépésköz pedig legyen \[h = \frac{b-a}{N} = t_i - t_{i-1} \]

Használjuk fel a Taylor sorbafejtést másodrendben x-re.
\begin{equation*}
    x(t_{i+1}) = x(t_i) + h x'(t_i) + \frac{h^2}{2} x''(\xi_i) \;\;,\;ahol \;\;\xi _i \in (t_i, t_{i+1})
\end{equation*}

Ezek alapján pedig láthatjuk, hogy az algoritmust a következőképpen írhatjuk fel:
\begin{equation*}
    \begin{cases}
        x_0 = \alpha \\
        x_{i+1} = x_i + h f(t_i,x_i) \;\;\forall i \in{0,1,...,N-1}
    \end{cases}       
\end{equation*}
\subsection{RK4}
Runge - Kutta módszerekből sokat alkottak az utóbbi időkben. Mi most a Runge-Kutta negyedfokó közelítését használjuk, amire a szakirodalomban sokszor csak 'RK4' módon hivatkoznak. A 4-ed rendű közelítés a következő formulából vezethető le:
\begin{equation*}
\begin{split}
      y_{t+h} = y_t + h \sum_{i=1}^{s} a_i k_i + \mathcal{O}(h^{s+1}),\\ 
      ahol\;\;k_i = y_t + h \sum_{j=1}^{s} \beta_{i,j} f(k_j,t_n + \alpha_ih)  
\end{split}
\end{equation*}

A képlete pedig s=4 mellett:
\begin{equation*}
\begin{cases}
        x_0 = \alpha \\
        k_1 = h f(t_i,w_i) \\
        k_2 = h f(t_i + \frac{h}{2}, w_i + \frac{1}{2} k_1) \\
        k_3 = h f(t_{i + 1} + \frac{h}{2}, w_i + \frac{1}{2} k_2) \\
        k_4 = h f(t_{i + 1}, w_i + k_3)\\
        w_{i+1} = w_i + \frac{1}{6}(k_1 + 2k_2 + 2k_3 + k_4)\\
        t_{i+1} = \alpha + ih
\end{cases}
\end{equation*}
Ezt pedig már tudjuk iterálni i-ben egy rögzített h mellett. Ennek a közelítésnek előnye, hogy a hibahatára már elfogadható és nem igényel annyira sok számítást, mint egy magasabb rendű közelítés
\subsection{Példa feladatok}
\begin{example} [Konstans együtthatós lineáris ODE]
Legyen \\
\begin{center}
   \begin{cases}
    x' = 5x - 3 \\
    x(0) = 1
   \end{cases} 
\end{center}
Ennek a kezdetiérték feladatnak az analitikus megoldása a következő:
\begin{equation*}
\begin{split}
        \frac{dx}{dt} &= 5x - 3 \\
        \frac{1}{5x-3} dx &= 1\; dt \\
        \int \frac{1}{5x-3} dx &= \int 1\; dt \\
        \frac{1}{5} ln|5x-3| &=t + C \\
        5x -3 &=e^{5(t + C)} \\
        x &= \frac{e^{5(t+C)} + 3}{5}
\end{split}
\end{equation*}
Mivel x(0) = 1, így
\begin{equation*}
    x(0) = \frac{e^{0+C} + 3}{5} = \frac{e^{5C}+3}{5} = 1
\end{equation*}
\begin{equation*}
    e^{5C} + 3 = 5
\end{equation*}
\begin{equation*}
    5C = ln2
\end{equation*}
\begin{equation*}
    C = \frac{ln2}{5}
\end{equation*}
Tehát a keresett függvény :
\begin{equation*}
    x(t) = \frac{e^{5(t + \frac{ln2}{5})} + 3}{5}
\end{equation*}
Nézzük meg mennyit ad a függvény a t=1 pontban:
\begin{equation*}
    x(1) = \frac{e^{5(1 + \frac{ln2}{5})} + 3}{5} \approx 59.9653
\end{equation*}
A függvényünk h = 0.1-es lépésköz mellett 59.8632 -es értéket adott. Ha h = 0.01, akkor sok tizedesjegyig megegyezik az eredmény. Azonban nem feltétlen szabad a megoldó algoritmusban h - t igazán kicsinek venni, mert könnyen a visszajára fordulhat. (Ez például függ attól, hogy hány bites a processzor. Nagyon nagy irodalma van az optimális értékek beállítására, azonban most ez a szakdolgozat ezzel nem foglalkozik)
\end{example}

\begin{example}[Folytonos kamatozás]
Tegyük fel, hogy elhelyezünk egy bankban konstans kamatláb mellett egy betétet folytonos kamatozás mellett. Tudjuk, hogy a folytonos kamatozás képlete
\begin{equation*}
    e^{rt},\;\; ahol \; r\; a\; kamatláb \;és\;t\; az \;idő
\end{equation*}
Ezt kifejezhetjük egy kezdetiérték probléma alakban is a következő módon : \\
\begin{center}
  \begin{cases}
        x' = rx \\
        x(0) = 1
  \end{cases}  
\end{center}
Ennek a megoldása pedig pont visszaadja a fentebb írt képletet: \\
\begin{equation*}
    \frac{dx}{dt} = r x
\end{equation*}
\begin{equation*}
    \frac{1}{x} dx = r dt
\end{equation*}
\begin{equation*}
    \int \frac{1}{x} dx = \int r dt
\end{equation*}
\begin{equation*}
    ln|x| = rt + c
\end{equation*}
\begin{equation*}
    x = e^{rt + c}
\end{equation*}
Mivel a kezdetifeltétel x(0) = 1, így c = 0 és visszakapjuk az eredeti képletet\\

Nézzük meg mennyi lehet x(1), úgy hogy r = 0.1 a kamatláb. A képlet szerint tudjuk, hogy
\begin{equation*}
    e^0.1 = x(1) = 1.105170918...
\end{equation*}
Ha ezt az értéket beheleyttesítjük a numerikus RK4-es függvényünkbe, akkor az x(1) = 1.1051709180665144 eredményt kapjuk 0.1-es lépésköz mellett. Ennél példánál a negyedrendű Runge-Kutta módszer elég pontos eredményt adott
\end{example}


\begin{example}[Másik példa]\\
Legyen \\
\begin{center}
   \begin{cases}
    x' = \frac{x^3}{3} \\
    x(0) = 3
   \end{cases} 
\end{center}
egy kezdeti érték feladat. Ennek a megoldása pedig
\begin{equation*}
    x(t) = \frac{3}{\sqrt{1-6t}}
\end{equation*}

Az analitikus eredmény t = 0.1 pontban :
\begin{equation*}
    x(0.1) \approx 4.7434
\end{equation*}
A Runge-Kutta módszer eredménye úgyszintén 4.7434.
\end{example}

\begin{example} [Egy nehezebb elsőrendű feladat]
Nézzük a következő feladatot: \\
\begin{center}
    \begin{cases}
     x' = \frac{1}{x(9 + 4 t^2)} \\
     x(0) = 1
    \end{cases}
\end{center}    
Ennek a feladatnak az analitikus megoldása :
\begin{equation*}
    x(t) = \sqrt{\frac{1}{3} arctg(\frac{2t}{3}) + 1}
\end{equation*}
\begin{equation*}
   x(0) = \sqrt{\frac{1}{3} arctg(\frac{0}{3}) + 1} = 1 
\end{equation*}
A numerikus eredmény pedig: 1.0110.
Ez az első feladat, ahol jelentős az eltérés, nevesen kb. 0.01
\end{example}
A Runge-Kutta és minden egyéb numerikus módszer tehát nem arra jó, hogy meghatározzuk analitikusan , képlettel a függvény formáját, hanem egy közelítést ad arra, hogy mi lehet a keresett függvény \textbf{értéke az adott pontban.}
\subsection{Python kódok a fejezethez}
\lstinputlisting[language=Python]{RK4/rk4.py}
\section{Stabilitás elméleti kitekintés}
A stabilitás elmélet egy nagy része a differenciálegyenletek témakörének. Ebben a dolgozatban szeretnénk egy kicsi betekintést nyerni ebbe a világba. Első körben tisztázzunk néhány definíciót
\subsection{Alapfogalmak}
A fejezetben végig vegyük adottnak a 
\begin{equation*}
    x' = f(t,x)
\end{equation*}
függvényt, ahol 
\begin{equation*}
    f: I \times \Omega \rightarrow \mathbb{R}^n \\
\end{equation*}
\begin{equation*}
    (t,x) \rightarrow f(t,x)
\end{equation*}
\begin{equation*}
    I=(t,\infty)
\end{equation*}
\begin{equation*}
    0 \in I
\end{equation*}
\begin{definition}[Egyensúlyi pont]
A fent bevezetett függvény mellett az
\begin{equation*}
    x' = 0 = f(t,x)
\end{equation*}
pontot egyensúlyi pontnak nevezzük
\end{definition}
A továbbiakban az 
\begin{equation*}
    (x_0,t_o) \in I \times \Omega 
\end{equation*}
legyen olyan intervallum, hogy f-nek egy megoldása legyen csak

\begin{definition}
Jelöljük a következő módon a vektorok felsőbecslésének halmazát :
\begin{equation*}
    B_{\rho} : = \{ x \in \mathbb{R}^n: \lVert x \rVert < \rho \}
\end{equation*}
\end{definition}
A következő néhány definíció fontos alapfogalmai a stabilitáselméletnek, így minnél pontosabb megfogalmazásra törekedünk, a [Sárga könyv] alapján. Amikor egy stabilitás fajtát tárgyalunk, azt a megoldásra kell értelmezni. Tehát ha megállapítjuk, hogy valami stabil, akkor azt arra fogjuk érteni, hogy a megoldás stabil.
\begin{definition} [Stabilitás]
\begin{equation*}
    \forall \epsilon > 0 \;-hoz\;\;és\;\;t_0 \in I \;-re\; \exists\;
    \delta >0\;,\;hogy\;\; \forall x_0 \in B_{\delta} \;és\; t \in J^+ \;esetén
\end{equation*}
\begin{equation*}
    \lVert x(t,(t_0,x_0)) \rVert < \epsilon
\end{equation*}
\end{definition}

\begin{definition}[Instabilitás]
Az instabil, ami nem stabil.
\end{definition}

\begin{definition}[Vonzás (attraktivitás)]
Az x' = f(t,x) x=0 megoldását vonzónak nevezzük, ha
\begin{equation*}
    \forall t_0 \in I \; esetén\; \exists\; \eta = \eta(t_0)
\end{equation*}
\begin{center}
    és
\end{center}
\begin{equation*}
    \forall \epsilon > 0\; esetén \;és\; \forall \; \lVert x_0 \rVert < \eta \;-ra \; \exists \; \sigma = \sigma(t_0,\epsilon,x_0) > 0
\end{equation*}
\begin{center}
    olyan módon, hogy
\end{center}
\begin{equation*}
    \forall \; t_0 + \sigma \leq t \;és\; t \in J^+ \;esetén \; \lVert x(t,t_o,x_0) \rVert < \epsilon
\end{equation*}
\end{definition}
\begin{definition}[Egyenletes stabilitás]
\begin{equation*}
    \forall \; \epsilon > 0 \; mellett \; \exists \; \delta  = \delta(\epsilon) \;olyan \;módon,\;hogy
\end{equation*}
\begin{equation*}
    \lVert x(t,(t_0,x_0)) \rVert < \epsilon
\end{equation*}
\begin{equation*}
    \forall t_0 \in I, \forall \; \lVert x_0 \rVert < \delta,
    \forall \; t_0 \leq t-ra \;teljesül
\end{equation*}
\end{definition}

\begin{definition}[Egyenlően vonzó]
\begin{equation*}
    \forall \; t_0 \in I \; esetén \; \exists \eta = \eta(t_0).
\end{equation*}
\begin{center}
    és
\end{center}
\begin{equation*}
    \forall \epsilon > 0 \; mellett \; \exists \; \sigma = \sigma(t_0,\epsilon) > 0
\end{equation*}
\begin{center}
    olyan módon, hogy
\end{center}
\begin{equation*}
    t_0 + \sigma \in J^+ : \lVert x(t,(t_0,x_0)) \rVert < \epsilon
    \; 
\end{equation*}
\begin{equation*}
     \forall \; \lVert x_0 \rVert < \eta
\end{equation*}
\begin{equation*}
     \forall t_0 + \sigma \leq ; \; t \in J^+
\end{equation*}
\end{definition}

\begin{definition}[Egyenletesen vonzó]
\begin{equation*}
  \exists \eta > 0 \;és \; \forall \; \epsilon > 0 \;mellett \; \exists \sigma = \sigma(\epsilon)   
\end{equation*}
\begin{center}
olyan módon, hogy $t_0 + \sigma \in J^+,$ és
\end{center}
\begin{equation*}
    \lVert x(t,(t_0,x_0)) \rVert < \epsilon
\end{equation*}
\begin{equation*}
    \forall \lVert x_0 \rVert < \eta, \;és
\end{equation*}
\begin{equation*}
    t_0 + \sigma \leq t, t \in J^+
\end{equation*}
\end{definition}

\begin{definition}[Globálisan vonzó]
\begin{equation*}
    \forall \eta > 0 \;és\; \forall t_0 \in I\;esetén
\end{equation*}
\begin{equation*}
     \lim_{t\to 0} x(t,(t_0,x_0))  \rightarrow 0
\end{equation*}
és emellé mind $t_0$-ban ,mind $x_0$-ban egyenletes a konvergencia
\end{definition}

A globálisan vonzó megoldás definíciója után ésszerű lehet kimondani ,hogy milyen lehet egy pont vonzási körzete, jelesül az origóé.

\begin{definition}[Origó vonzási körzete vagy attraktivitási tartománya]
Jelölje A() a következő halmazt :
\begin{equation*}
    A(t_0) := \{ x_0 \in \Omega : \lim_{t\to 0} x(t,(t_0,x_0))  \rightarrow 0 \}
\end{equation*}
Ha ez $t_0$-tól független, akkor egyenletesnek mondjuk a vonzási körzetet.
\end{definition}

Az alapvető definíciók mellé még mondjunk ki 4 megoldási jellemzhetőséget, amelyek egymásnak finomításai.

\begin{definition}[Aszimptotikus stabilitás]
Az x' = f(t,x) függvénynek az x = 0 megoldási aszimptotikusan stabil, ha stabil és vonzó
\end{definition}

\begin{definition}[Egyenlően aszimptotikusan stabil]
Minden megoldás egyenlően aszimptotikusan stabil, ha stabil és egyenlően vonzó
\end{definition}

\begin{definition}[Egyenletesen aszimptotikusan stabil]
Egyenletesen aszimptotikusan stabilnak nevezzük azokat a megoldásokat, amelyek egyenletesen stabilok és egyenletesen vonzóak
\end{definition}

\begin{definition}[Globálisan egyenletesen aszimptotikusan stabil]
Globálisan egyenletesen aszimptotikusan stabil hívjuk azokat a megoldásokat, amelyek egyenletesen stabil és egyenletesen globálisan vonzó
\end{definition}
\subsection{Rendszerek stabilitásának vizsgálata}
\section{Perturbáció elméleti megközelítés}
...Ezzel a résszel a stabilitás után fogok foglalkozni...
\section{Felhasznált irodalom}
\begin{itemize}
    \item Kis Ottó, Kovács Margit - Numerikus módszerek
    \item Richard L. Burden, J Douglas Faires - Numerical Analysis (9th edition)
    \item Kánnai Zoltán dinamika jegyzetei
    \item stabilitásos sárga könyv
\end{itemize}
\section{Numba}
\subsection{költségminimum}
egyensúly: akkor maximális a profit, ha a költség minimális -> ezen az analógián felvezetem a Numba-t, hogy ha nagy szimulációt végzek akkor érdemes numpyban megírni és ráereszteni a numba jit dekorátort + vanilla python így lassú, lassú de numpyos már, gyors mert numpy és numbás példákat beirni
\section{Git és GitHub}
A dolgozat tex fájlban készült és csatoltam mellé sok Python szkriptet. Mivel sok a kód, nehéz nyomon követni, hogy mikor mit változtattam meg, esetleg kitöröltem valami olyasmit, amire nem volt szükség vagy éppen kitöröltem olyat, amelyet nem kellett volna. A Git verziókövető nyelvvel könnyen el lehet menteni a kódokban (és fájlokban) a változtatásokat , és minden változtatáshoz egy üzenetet kell csatolnunk az átláthatóság érdekében. Mivel a Git csak a fájlokat mozgatja, nem feltétlen könnyű átlátni a terminálban, hogy épp milyen állapotban vannak a fájlok. Erre nyújt megoldást a GitHub, ahol egy felhasználóbarát interfészen keresztül látható mindaz, amit felsoroltam. A szakdolgozat teljes formájában elérhető ezen a linken keresztül :
\begin{center}
    https://github.com/SzabBence?tab=repositories
\end{center}
\end{document}
